\documentclass[journal,12pt,twocolumn]{IEEEtran}
%
\usepackage{setspace}
\usepackage{gensymb}
%\doublespacing
\singlespacing

%\usepackage{graphicx}
%\usepackage{amssymb}
%\usepackage{relsize}
\usepackage[cmex10]{amsmath}
%\usepackage{amsthm}
%\interdisplaylinepenalty=2500
%\savesymbol{iint}
%\usepackage{txfonts}
%\restoresymbol{TXF}{iint}
%\usepackage{wasysym}
\usepackage{amsthm}
%\usepackage{iithtlc}
\usepackage{mathrsfs}
\usepackage{txfonts}
\usepackage{stfloats}
\usepackage{bm}
\usepackage{cite}
\usepackage{cases}
\usepackage{subfig}
%\usepackage{xtab}
\usepackage{longtable}
\usepackage{multirow}
%\usepackage{algorithm}
%\usepackage{algpseudocode}
\usepackage{enumitem}
\usepackage{mathtools}
\usepackage{steinmetz}
\usepackage{tikz}
\usepackage{circuitikz}
\usepackage{verbatim}
\usepackage{tfrupee}
\usepackage[breaklinks=true]{hyperref}
%\usepackage{stmaryrd}
\usepackage{tkz-euclide} % loads  TikZ and tkz-base
%\usetkzobj{all}
\usetikzlibrary{calc,math}
\usepackage{listings}
    \usepackage{color}                                            %%
    \usepackage{array}                                            %%
    \usepackage{longtable}                                        %%
    \usepackage{calc}                                             %%
    \usepackage{multirow}                                         %%
    \usepackage{hhline}                                           %%
    \usepackage{ifthen}                                           %%
  %optionally (for landscape tables embedded in another document): %%
    \usepackage{lscape}     
\usepackage{multicol}
\usepackage{chngcntr}
%\usepackage{enumerate}

%\usepackage{wasysym}
%\newcounter{MYtempeqncnt}
\DeclareMathOperator*{\Res}{Res}
%\renewcommand{\baselinestretch}{2}
\renewcommand\thesection{\arabic{section}}
\renewcommand\thesubsection{\thesection.\arabic{subsection}}
\renewcommand\thesubsubsection{\thesubsection.\arabic{subsubsection}}

\renewcommand\thesectiondis{\arabic{section}}
\renewcommand\thesubsectiondis{\thesectiondis.\arabic{subsection}}
\renewcommand\thesubsubsectiondis{\thesubsectiondis.\arabic{subsubsection}}

% correct bad hyphenation here
\hyphenation{op-tical net-works semi-conduc-tor}
\def\inputGnumericTable{}                                 %%

\lstset{
%language=C,
frame=single, 
breaklines=true,
columns=fullflexible
}
%\lstset{
%language=tex,
%frame=single, 
%breaklines=true
%}

\begin{document}
%
\newtheorem{theorem}{Theorem}[section]
\newtheorem{problem}{Problem}
\newtheorem{proposition}{Proposition}[section]
\newtheorem{lemma}{Lemma}[section]
\newtheorem{corollary}[theorem]{Corollary}
\newtheorem{example}{Example}[section]
\newtheorem{definition}[problem]{Definition}
%\newtheorem{thm}{Theorem}[section] 
%\newtheorem{defn}[thm]{Definition}
%\newtheorem{algorithm}{Algorithm}[section]
%\newtheorem{cor}{Corollary}
\newcommand{\BEQA}{\begin{eqnarray}}
\newcommand{\EEQA}{\end{eqnarray}}
\newcommand{\define}{\stackrel{\triangle}{=}}
\bibliographystyle{IEEEtran}
%\bibliographystyle{ieeetr}
\providecommand{\mbf}{\mathbf}
\providecommand{\pr}[1]{\ensuremath{\Pr\left(#1\right)}}
\providecommand{\qfunc}[1]{\ensuremath{Q\left(#1\right)}}
\providecommand{\sbrak}[1]{\ensuremath{{}\left[#1\right]}}
\providecommand{\lsbrak}[1]{\ensuremath{{}\left[#1\right.}}
\providecommand{\rsbrak}[1]{\ensuremath{{}\left.#1\right]}}
\providecommand{\brak}[1]{\ensuremath{\left(#1\right)}}
\providecommand{\lbrak}[1]{\ensuremath{\left(#1\right.}}
\providecommand{\rbrak}[1]{\ensuremath{\left.#1\right)}}
\providecommand{\cbrak}[1]{\ensuremath{\left\{#1\right\}}}
\providecommand{\lcbrak}[1]{\ensuremath{\left\{#1\right.}}
\providecommand{\rcbrak}[1]{\ensuremath{\left.#1\right\}}}
\theoremstyle{remark}
\newtheorem{rem}{Remark}
\newcommand{\sgn}{\mathop{\mathrm{sgn}}}
\providecommand{\abs}[1]{\left\vert#1\right\vert}
\providecommand{\res}[1]{\Res\displaylimits_{#1}} 
\providecommand{\norm}[1]{\left\lVert#1\right\rVert}
%\providecommand{\norm}[1]{\lVert#1\rVert}
\providecommand{\mtx}[1]{\mathbf{#1}}
\providecommand{\mean}[1]{E\left[ #1 \right]}
\providecommand{\fourier}{\overset{\mathcal{F}}{ \rightleftharpoons}}
%\providecommand{\hilbert}{\overset{\mathcal{H}}{ \rightleftharpoons}}
\providecommand{\system}{\overset{\mathcal{H}}{ \longleftrightarrow}}
	%\newcommand{\solution}[2]{\textbf{Solution:}{#1}}
\newcommand{\solution}{\noindent \textbf{Solution: }}
\newcommand{\cosec}{\,\text{cosec}\,}
\providecommand{\dec}[2]{\ensuremath{\overset{#1}{\underset{#2}{\gtrless}}}}
\newcommand{\myvec}[1]{\ensuremath{\begin{pmatrix}#1\end{pmatrix}}}
\newcommand{\mydet}[1]{\ensuremath{\begin{vmatrix}#1\end{vmatrix}}}
\newcommand\inv[1]{#1\raisebox{1.15ex}{$\scriptscriptstyle-\!1$}}

%\numberwithin{equation}{section}
\numberwithin{equation}{subsection}
%\numberwithin{problem}{section}
%\numberwithin{definition}{section}
\makeatletter
\@addtoreset{figure}{problem}
\makeatother
\let\StandardTheFigure\thefigure
\let\vec\mathbf
%\renewcommand{\thefigure}{\theproblem.\arabic{figure}}
\renewcommand{\thefigure}{\theproblem}
%\setlist[enumerate,1]{before=\renewcommand\theequation{\theenumi.\arabic{equation}}
%\counterwithin{equation}{enumi}
%\renewcommand{\theequation}{\arabic{subsection}.\arabic{equation}}
\def\putbox#1#2#3{\makebox[0in][l]{\makebox[#1][l]{}\raisebox{\baselineskip}[0in][0in]{\raisebox{#2}[0in][0in]{#3}}}}
     \def\rightbox#1{\makebox[0in][r]{#1}}
     \def\centbox#1{\makebox[0in]{#1}}
     \def\topbox#1{\raisebox{-\baselineskip}[0in][0in]{#1}}
     \def\midbox#1{\raisebox{-0.5\baselineskip}[0in][0in]{#1}}
\vspace{3cm}
\title{EE5609: Matrix Theory\\
          Assignment-8\\}
\author{M Pavan Manesh\\
EE20MTECH14017 }
\maketitle
\newpage
%\tableofcontents
\bigskip
\renewcommand{\thefigure}{\theenumi}
\renewcommand{\thetable}{\theenumi}
\begin{abstract}
This document explains how to find the basis for the given vector space
\end{abstract}
Download all latex-tikz codes from 
%
\begin{lstlisting}
https://github.com/pavanmanesh/EE5609/tree/master/Assignment
\end{lstlisting}
%
\section{Problem}
Let $\mathbb{V}$ be a vector space which is spanned by the rows of matrix
\begin{align}
    \vec{A} = \myvec{3&21&0&9&0\\1&7&-1&-2&-1\\2&14&6&0&1\\6&42&-1&13&0}\label{eq:mat}
\end{align}
\begin{enumerate}[label=\alph*.]
\item Find a basis for $\mathbb{V}$
\item Tell which vectors \myvec{x_1&x_2&x_3&x_4&x_5} are elements of $\mathbb{V}$ 
\item If \myvec{x_1&x_2&x_3&x_4&x_5} is in $\mathbb{V}$ ,what are its coordinates in the basis chosen?
\end{enumerate}
\section{Solution}
Row reducing \eqref{eq:mat}
\begin{align}
\myvec{3&21&0&9&0\\1&7&-1&-2&-1\\2&14&6&0&1\\6&42&-1&13&0} \nonumber\\
\xleftrightarrow[]{R_1\leftarrow\frac{R_1}{3}}
\myvec{1&7&0&3&0\\1&7&-1&-2&-1\\2&14&6&0&1\\6&42&-1&13&0} \nonumber \\
\xleftrightarrow[R_2\leftarrow R_2-R_1]{R_3\leftarrow R_3-2R_1}\myvec{1&7&0&3&0\\0&0&-1&-5&-1\\0&0&0&0&1\\0&0&-1&-5&0}  \nonumber\\
\xleftrightarrow[]{R_4\leftarrow R_4-R_2}\myvec{1&7&0&3&0\\0&0&-1&-5&-1\\0&0&0&0&1\\0&0&0&0&1}  \nonumber\\
\xleftrightarrow[]{R_2\leftarrow -R_2}\myvec{1&7&0&3&0\\0&0&1&5&1\\0&0&0&0&1\\0&0&0&0&1}  \nonumber\\
\xleftrightarrow[R_2\leftarrow R_2-R_3]{R_4\leftarrow R_4-R_3}\myvec{1&7&0&3&0\\0&0&1&5&1\\0&0&0&0&1\\0&0&0&0&0} \label{eq:rref}
\end{align}
\begin{enumerate}[label=\alph*.]
\item For the basis of $\mathbb{V}$, we can take the non zero rows of \eqref{eq:rref} \\
\begin{align}
    \rho_1=\myvec{1&7&0&3&0} \\
    \rho_2=\myvec{0&0&1&5&1}\\
    \rho_3=\myvec{0&0&0&0&1}
\end{align}
\item Vectors which are elements of $\mathbb{V}$  are of the form
\begin{align}
    c_1\rho_1+c_2\rho_2+c_3\rho_3 \nonumber\\
    =\myvec{c_1&7c_1&c_2&3c_1+5c_2&c_3} \label{eq:vec}
\end{align}
where $c_1$,$c_2$,$c_3$ are scalars.
\item By \eqref{eq:vec},if \myvec{x_1&x_2&x_3&x_4&x_5} is in $\mathbb{V}$,it must be of the form
\begin{align}
    x_1\rho_1+x_3\rho_2+x_5\rho_3
\end{align}
The coordinates of \myvec{x_1&x_2&x_3&x_4&x_5} in the  basis is
\begin{align}
    \myvec{x_1\\x_3\\x_5}
\end{align}
\end{enumerate}
\end{document}
