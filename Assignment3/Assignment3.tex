\documentclass[journal,12pt,twocolumn]{IEEEtran}
%
\usepackage{setspace}
\usepackage{gensymb}
\usepackage{siunitx}
\usepackage{tkz-euclide} 
\usepackage{textcomp}
\usepackage{standalone}
\usetikzlibrary{calc}
\newcommand\hmmax{0}
\newcommand\bmmax{0}

%\doublespacing
\singlespacing

%\usepackage{graphicx}
%\usepackage{amssymb}
%\usepackage{relsize}
\usepackage[cmex10]{amsmath}
%\usepackage{amsthm}
%\interdisplaylinepenalty=2500
%\savesymbol{iint}
%\usepackage{txfonts}
%\restoresymbol{TXF}{iint}
%\usepackage{wasysym}
\usepackage{amsthm}
%\usepackage{iithtlc}
\usepackage{mathrsfs}
\usepackage{txfonts}
\usepackage{stfloats}
\usepackage{bm}
\usepackage{cite}
\usepackage{cases}
\usepackage{subfig}
%\usepackage{xtab}
\usepackage{longtable}
\usepackage{multirow}
%\usepackage{algorithm}
%\usepackage{algpseudocode}
\usepackage{enumitem}
\usepackage{mathtools}
\usepackage{steinmetz}
\usepackage{tikz}
\usepackage{circuitikz}
\usepackage{verbatim}
\usepackage{tfrupee}
\usepackage[breaklinks=true]{hyperref}
%\usepackage{stmaryrd}
\usepackage{tkz-euclide} % loads  TikZ and tkz-base
%\usetkzobj{all}
\usetikzlibrary{calc,math}
\usepackage{listings}
    \usepackage{color}                                            %%
    \usepackage{array}                                            %%
    \usepackage{longtable}                                        %%
    \usepackage{calc}                                             %%
    \usepackage{multirow}                                         %%
    \usepackage{hhline}                                           %%
    \usepackage{ifthen}                                           %%
  %optionally (for landscape tables embedded in another document): %%
    \usepackage{lscape}     
\usepackage{multicol}
\usepackage{chngcntr}
\usepackage{amsmath}
\usepackage{cleveref}
%\usepackage{enumerate}

%\usepackage{wasysym}
%\newcounter{MYtempeqncnt}
\DeclareMathOperator*{\Res}{Res}
%\renewcommand{\baselinestretch}{2}
\renewcommand\thesection{\arabic{section}}
\renewcommand\thesubsection{\thesection.\arabic{subsection}}
\renewcommand\thesubsubsection{\thesubsection.\arabic{subsubsection}}

\renewcommand\thesectiondis{\arabic{section}}
\renewcommand\thesubsectiondis{\thesectiondis.\arabic{subsection}}
\renewcommand\thesubsubsectiondis{\thesubsectiondis.\arabic{subsubsection}}

% correct bad hyphenation here
\hyphenation{op-tical net-works semi-conduc-tor}
\def\inputGnumericTable{}                                 %%

\lstset{
%language=C,
frame=single, 
breaklines=true,
columns=fullflexible
}
%\lstset{
%language=tex,
%frame=single, 
%breaklines=true
%}
\usepackage{graphicx}
\usepackage{pgfplots}

\begin{document}


\newtheorem{theorem}{Theorem}[section]
\newtheorem{problem}{Problem}
\newtheorem{proposition}{Proposition}[section]
\newtheorem{lemma}{Lemma}[section]
\newtheorem{corollary}[theorem]{Corollary}
\newtheorem{example}{Example}[section]
\newtheorem{definition}[problem]{Definition}
%\newtheorem{thm}{Theorem}[section] 
%\newtheorem{defn}[thm]{Definition}
%\newtheorem{algorithm}{Algorithm}[section]
%\newtheorem{cor}{Corollary}
\newcommand{\BEQA}{\begin{eqnarray}}
\newcommand{\EEQA}{\end{eqnarray}}
\newcommand{\define}{\stackrel{\triangle}{=}}
\bibliographystyle{IEEEtran}
%\bibliographystyle{ieeetr}
\providecommand{\mbf}{\mathbf}
\providecommand{\abs}[1]{\ensuremath{\left\vert#1\right\vert}}
\providecommand{\norm}[1]{\ensuremath{\left\lVert#1\right\rVert}}
\providecommand{\mean}[1]{\ensuremath{E\left[ #1 \right]}}
\providecommand{\pr}[1]{\ensuremath{\Pr\left(#1\right)}}
\providecommand{\qfunc}[1]{\ensuremath{Q\left(#1\right)}}
\providecommand{\sbrak}[1]{\ensuremath{{}\left[#1\right]}}
\providecommand{\lsbrak}[1]{\ensuremath{{}\left[#1\right.}}
\providecommand{\rsbrak}[1]{\ensuremath{{}\left.#1\right]}}
\providecommand{\brak}[1]{\ensuremath{\left(#1\right)}}
\providecommand{\lbrak}[1]{\ensuremath{\left(#1\right.}}
\providecommand{\rbrak}[1]{\ensuremath{\left.#1\right)}}
\providecommand{\cbrak}[1]{\ensuremath{\left\{#1\right\}}}
\providecommand{\lcbrak}[1]{\ensuremath{\left\{#1\right.}}
\providecommand{\rcbrak}[1]{\ensuremath{\left.#1\right\}}}
\theoremstyle{remark}
\newtheorem{rem}{Remark}
\newcommand{\sgn}{\mathop{\mathrm{sgn}}}
\providecommand{\res}[1]{\Res\displaylimits_{#1}} 
%\providecommand{\norm}[1]{\lVert#1\rVert}
\providecommand{\mtx}[1]{\mathbf{#1}}
\providecommand{\fourier}{\overset{\mathcal{F}}{ \rightleftharpoons}}
%\providecommand{\hilbert}{\overset{\mathcal{H}}{ \rightleftharpoons}}
\providecommand{\system}{\overset{\mathcal{H}}{ \longleftrightarrow}}
	%\newcommand{\solution}[2]{\textbf{Solution:}{#1}}
\newcommand{\solution}{\noindent \textbf{Solution: }}
\newcommand{\cosec}{\,\text{cosec}\,}
\providecommand{\dec}[2]{\ensuremath{\overset{#1}{\underset{#2}{\gtrless}}}}
\newcommand{\myvec}[1]{\ensuremath{\begin{pmatrix}#1\end{pmatrix}}}
\newcommand{\mydet}[1]{\ensuremath{\begin{vmatrix}#1\end{vmatrix}}}
%\numberwithin{equation}{section}
\numberwithin{equation}{subsection}
%\numberwithin{problem}{section}
%\numberwithin{definition}{section}
\makeatletter
\@addtoreset{figure}{problem}
\makeatother
\let\StandardTheFigure\thefigure
\let\vec\mathbf
%\renewcommand{\thefigure}{\theproblem.\arabic{figure}}
\renewcommand{\thefigure}{\theproblem}
%\setlist[enumerate,1]{before=\renewcommand\theequation{\theenumi.\arabic{equation}}
%\counterwithin{equation}{enumi}
%\renewcommand{\theequation}{\arabic{subsection}.\arabic{equation}}
\def\putbox#1#2#3{\makebox[0in][l]{\makebox[#1][l]{}\raisebox{\baselineskip}[0in][0in]{\raisebox{#2}[0in][0in]{#3}}}}
     \def\rightbox#1{\makebox[0in][r]{#1}}
     \def\centbox#1{\makebox[0in]{#1}}
     \def\topbox#1{\raisebox{-\baselineskip}[0in][0in]{#1}}}
\vspace{3cm}
\title{EE5609: Matrix Theory\\
          Assignment-3\\}
\author{M Pavan Manesh\\
EE20MTECH14017 }
\maketitle
\newpage
%\tableofcontents
\bigskip
\renewcommand{\thefigure}{\theenumi}
\renewcommand{\thetable}{\theenumi}
\begin{abstract}
This document contains a solution for showing that line AP bisects
\end{abstract}
Download the python codes from 
%
%
latex-tikz codes from 
%
\begin{lstlisting}
https://github.com/pavanmanesh/EE5609/tree/master/Assignment3
\end{lstlisting}
%
\section{PROBLEM}
P is a point equidistant from two lines l and m
intersecting at point A. Show that the line AP
bisects the angle between them.
\section{SOLUTION}
 \begin{figure}[!ht]
\centering
\resizebox{\columnwidth}{!}{\begin{tikzpicture} 
        \coordinate (B) at (2.5, -2.5) {};
        \coordinate (A) at (0, 0) {};
        \coordinate (P) at (5, 0) {};
        \coordinate (C) at (2.5, 2.5) {};

        \draw (C)node[above]{$C$}--(A)node[below]{$A$}--(P)node[above]{$P$}--cycle;
        \draw (P)node[above]{$P$}--(A)node[below]{$A$}--(B)node[below]{$B$}--cycle;
\tkzMarkRightAngle[size=.2](P,C,A);
\tkzLabelAngle[dist=.5](P,A,B){};
\tkzMarkRightAngle[size=.2](P,B,A);
\tkzLabelAngle[dist=.5](P,A,C){};
\end{tikzpicture}
}
\caption{figure}
\label{fig1}
\end{figure}
\begin{enumerate}
    \item Here, the following information is given:
    \begin{align}
    \norm{\vec{P}-\vec{B}}=\norm{\vec{P}-\vec{C}} \label{eq2.1}
    \end{align}
    \item The lines $PB$ is the perpendicular to line $AB$ and 
    $PC$ is the perpendicular to line $AC$:
    \begin{align}
    \myvec{\vec{P}-\vec{B}}^T\myvec{\vec{A}-\vec{B}}=0 \label{eq2.2}
    \implies \cos\angle PBA=0 
    \end{align}
    \begin{align}
    \myvec{\vec{P}-\vec{C}}^T\myvec{\vec{A}-\vec{C}}=0 \label{eq2.3}
    \implies \cos\angle PCA=0 
    \end{align}
\end{enumerate}
We know that 
\begin{align}
\norm{\vec{P-A}}^2=(\vec{P}-\vec{A})^T(\vec{P}-\vec{A})
\end{align}
\begin{align}
 \begin{split}
(\vec{P}-\vec{A})^T(\vec{P}-\vec{A})=(\vec{P}-\vec{B}+\vec{B}-\vec{A})^T(\vec{P}-\vec{B}+\vec{B}-\vec{A})
\end{split}
\end{align}
\begin{align}
 \begin{split}
\norm{\vec{P-A}}^2=\norm{\vec{P-B}}^2+\norm{\vec{B-A}}^2\\ 
+2\norm{\vec{A}-\vec{P}}\norm{\vec{B}-\vec{A}}\cos\angle PBA \\
\end{split}
\end{align}
using \eqref{eq2.2}
\begin{align}
 \begin{split}
 \implies \norm{\vec{P-A}}^2=\norm{\vec{P-B}}^2+\norm{\vec{B-A}}^2 \label{eq2.4} \\
 \end{split}
\end{align}
Similarly
\begin{align}
 \begin{split}
(\vec{P}-\vec{A})^T(\vec{P}-\vec{A})=(\vec{P}-\vec{C}+\vec{C}-\vec{A})^T(\vec{P}-\vec{C}+\vec{C}-\vec{A})
\end{split}
\end{align}
\begin{align}
 \begin{split}
\norm{\vec{P-A}}^2=\norm{\vec{P-C}}^2+\norm{\vec{C-A}}^2\\ 
+2\norm{\vec{A}-\vec{P}}\norm{\vec{C}-\vec{A}}\cos\angle PCA \\
\end{split}
\end{align}
using \eqref{eq2.3}
\begin{align}
 \begin{split}
 \implies \norm{\vec{P-A}}^2=\norm{\vec{P-C}}^2+\norm{\vec{C-A}}^2 \label{eq2.5}
 \end{split}
\end{align}
From \eqref{eq2.4} and \eqref{eq2.5} and substituting \eqref{eq2.1}
\begin{align}
\norm{\vec{P-C}}^2+\norm{\vec{C-A}}^2=\norm{\vec{P-B}}^2+\norm{\vec{B-A}}^2 \\
\implies \norm{\vec{C-A}}^2=\norm{\vec{B-A}}^2
\implies \norm{\vec{C-A}}=\norm{\vec{B-A}} \label{eq2.6}
\end{align}
We know that 
\begin{align}
\norm{\vec{P-B}}^2=(\vec{P}-\vec{B})^T(\vec{P}-\vec{B})
\end{align}
\begin{align}
 \begin{split}
(\vec{P}-\vec{B})^T(\vec{P}-\vec{B})=(\vec{P}-\vec{A}+\vec{A}-\vec{B})^T(\vec{P}-\vec{A}+\vec{A}-\vec{B})
\end{split}
\end{align}
\begin{align}
 \begin{split}
\norm{\vec{P-B}}^2=\norm{\vec{P-A}}^2+\norm{\vec{A-B}}^2\\ 
+2\norm{\vec{P}-\vec{A}}\norm{\vec{A}-\vec{B}}\cos\angle PAB \\ \label{eq2.7}
\end{split}
\end{align}
Similarly
\begin{align}
\norm{\vec{P-C}}^2=(\vec{P}-\vec{C})^T(\vec{P}-\vec{C})
\end{align}
\begin{align}
 \begin{split}
(\vec{P}-\vec{C})^T(\vec{P}-\vec{C})=(\vec{P}-\vec{A}+\vec{A}-\vec{C})^T(\vec{P}-\vec{A}+\vec{A}-\vec{C})
\end{split}
\end{align}
\begin{align}
 \begin{split}
\norm{\vec{P-C}}^2=\norm{\vec{P-A}}^2+\norm{\vec{A-C}}^2\\ 
+2\norm{\vec{P}-\vec{A}}\norm{\vec{A}-\vec{C}}\cos\angle PAC \\ \label{eq2.8}
\end{split}
\end{align}
From \eqref{eq2.1},Equating \eqref{eq2.7} and \eqref{eq2.8}
\begin{align}
\begin{split}
\norm{\vec{P-A}}^2+\norm{\vec{A-B}}^2 
+2\norm{\vec{P}-\vec{A}}\norm{\vec{A}-\vec{B}}\cos\angle PAB =\\
\norm{\vec{P-A}}^2+\norm{\vec{A-C}}^2
+2\norm{\vec{P}-\vec{A}}\norm{\vec{A}-\vec{C}}\cos\angle PAC \\
\end{split}
\end{align}
Using  \eqref{eq2.6} 
\begin{align}
\begin{split}
2\norm{\vec{P}-\vec{A}}\norm{\vec{A}-\vec{B}}\cos\angle PAB =\\
2\norm{\vec{P}-\vec{A}}\norm{\vec{A}-\vec{C}}\cos\angle PAC \\
\implies \cos\angle PAB = \cos\angle PAC
\end{split}
\end{align}
The line AP bisects the angle between them.
\end{document}
