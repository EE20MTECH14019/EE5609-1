\documentclass[journal,12pt,twocolumn]{IEEEtran}
%
\usepackage{setspace}
\usepackage{gensymb}
%\doublespacing
\singlespacing

%\usepackage{graphicx}
%\usepackage{amssymb}
%\usepackage{relsize}
\usepackage[cmex10]{amsmath}
%\usepackage{amsthm}
%\interdisplaylinepenalty=2500
%\savesymbol{iint}
%\usepackage{txfonts}
%\restoresymbol{TXF}{iint}
%\usepackage{wasysym}
\usepackage{amsthm}
%\usepackage{iithtlc}
\usepackage{mathrsfs}
\usepackage{txfonts}
\usepackage{stfloats}
\usepackage{bm}
\usepackage{cite}
\usepackage{cases}
\usepackage{subfig}
%\usepackage{xtab}
\usepackage{longtable}
\usepackage{multirow}
%\usepackage{algorithm}
%\usepackage{algpseudocode}
\usepackage{enumitem}
\usepackage{mathtools}
\usepackage{steinmetz}
\usepackage{tikz}
\usepackage{circuitikz}
\usepackage{verbatim}
\usepackage{tfrupee}
\usepackage[breaklinks=true]{hyperref}
%\usepackage{stmaryrd}
\usepackage{tkz-euclide} % loads  TikZ and tkz-base
%\usetkzobj{all}
\usetikzlibrary{calc,math}
\usepackage{listings}
    \usepackage{color}                                            %%
    \usepackage{array}                                            %%
    \usepackage{longtable}                                        %%
    \usepackage{calc}                                             %%
    \usepackage{multirow}                                         %%
    \usepackage{hhline}                                           %%
    \usepackage{ifthen}                                           %%
  %optionally (for landscape tables embedded in another document): %%
    \usepackage{lscape}     
\usepackage{multicol}
\usepackage{chngcntr}
%\usepackage{enumerate}

%\usepackage{wasysym}
%\newcounter{MYtempeqncnt}
\DeclareMathOperator*{\Res}{Res}
%\renewcommand{\baselinestretch}{2}
\renewcommand\thesection{\arabic{section}}
\renewcommand\thesubsection{\thesection.\arabic{subsection}}
\renewcommand\thesubsubsection{\thesubsection.\arabic{subsubsection}}

\renewcommand\thesectiondis{\arabic{section}}
\renewcommand\thesubsectiondis{\thesectiondis.\arabic{subsection}}
\renewcommand\thesubsubsectiondis{\thesubsectiondis.\arabic{subsubsection}}

% correct bad hyphenation here
\hyphenation{op-tical net-works semi-conduc-tor}
\def\inputGnumericTable{}                                 %%

\lstset{
%language=C,
frame=single, 
breaklines=true,
columns=fullflexible
}
%\lstset{
%language=tex,
%frame=single, 
%breaklines=true
%}

\begin{document}
%
\newtheorem{theorem}{Theorem}[section]
\newtheorem{problem}{Problem}
\newtheorem{proposition}{Proposition}[section]
\newtheorem{lemma}{Lemma}[section]
\newtheorem{corollary}[theorem]{Corollary}
\newtheorem{example}{Example}[section]
\newtheorem{definition}[problem]{Definition}
%\newtheorem{thm}{Theorem}[section] 
%\newtheorem{defn}[thm]{Definition}
%\newtheorem{algorithm}{Algorithm}[section]
%\newtheorem{cor}{Corollary}
\newcommand{\BEQA}{\begin{eqnarray}}
\newcommand{\EEQA}{\end{eqnarray}}
\newcommand{\define}{\stackrel{\triangle}{=}}
\bibliographystyle{IEEEtran}
%\bibliographystyle{ieeetr}
\providecommand{\mbf}{\mathbf}
\providecommand{\pr}[1]{\ensuremath{\Pr\left(#1\right)}}
\providecommand{\qfunc}[1]{\ensuremath{Q\left(#1\right)}}
\providecommand{\sbrak}[1]{\ensuremath{{}\left[#1\right]}}
\providecommand{\lsbrak}[1]{\ensuremath{{}\left[#1\right.}}
\providecommand{\rsbrak}[1]{\ensuremath{{}\left.#1\right]}}
\providecommand{\brak}[1]{\ensuremath{\left(#1\right)}}
\providecommand{\lbrak}[1]{\ensuremath{\left(#1\right.}}
\providecommand{\rbrak}[1]{\ensuremath{\left.#1\right)}}
\providecommand{\cbrak}[1]{\ensuremath{\left\{#1\right\}}}
\providecommand{\lcbrak}[1]{\ensuremath{\left\{#1\right.}}
\providecommand{\rcbrak}[1]{\ensuremath{\left.#1\right\}}}
\theoremstyle{remark}
\newtheorem{rem}{Remark}
\newcommand{\sgn}{\mathop{\mathrm{sgn}}}
\providecommand{\abs}[1]{\left\vert#1\right\vert}
\providecommand{\res}[1]{\Res\displaylimits_{#1}} 
\providecommand{\norm}[1]{\left\lVert#1\right\rVert}
%\providecommand{\norm}[1]{\lVert#1\rVert}
\providecommand{\mtx}[1]{\mathbf{#1}}
\providecommand{\mean}[1]{E\left[ #1 \right]}
\providecommand{\fourier}{\overset{\mathcal{F}}{ \rightleftharpoons}}
%\providecommand{\hilbert}{\overset{\mathcal{H}}{ \rightleftharpoons}}
\providecommand{\system}{\overset{\mathcal{H}}{ \longleftrightarrow}}
	%\newcommand{\solution}[2]{\textbf{Solution:}{#1}}
\newcommand{\solution}{\noindent \textbf{Solution: }}
\newcommand{\cosec}{\,\text{cosec}\,}
\providecommand{\dec}[2]{\ensuremath{\overset{#1}{\underset{#2}{\gtrless}}}}
\newcommand{\myvec}[1]{\ensuremath{\begin{pmatrix}#1\end{pmatrix}}}
\newcommand{\mydet}[1]{\ensuremath{\begin{vmatrix}#1\end{vmatrix}}}
\newcommand\inv[1]{#1\raisebox{1.15ex}{$\scriptscriptstyle-\!1$}}

%\numberwithin{equation}{section}
\numberwithin{equation}{subsection}
%\numberwithin{problem}{section}
%\numberwithin{definition}{section}
\makeatletter
\@addtoreset{figure}{problem}
\makeatother
\let\StandardTheFigure\thefigure
\let\vec\mathbf
%\renewcommand{\thefigure}{\theproblem.\arabic{figure}}
\renewcommand{\thefigure}{\theproblem}
%\setlist[enumerate,1]{before=\renewcommand\theequation{\theenumi.\arabic{equation}}
%\counterwithin{equation}{enumi}
%\renewcommand{\theequation}{\arabic{subsection}.\arabic{equation}}
\def\putbox#1#2#3{\makebox[0in][l]{\makebox[#1][l]{}\raisebox{\baselineskip}[0in][0in]{\raisebox{#2}[0in][0in]{#3}}}}
     \def\rightbox#1{\makebox[0in][r]{#1}}
     \def\centbox#1{\makebox[0in]{#1}}
     \def\topbox#1{\raisebox{-\baselineskip}[0in][0in]{#1}}
     \def\midbox#1{\raisebox{-0.5\baselineskip}[0in][0in]{#1}}
\vspace{3cm}
\title{EE5609: Matrix Theory\\
          Assignment-8\\}
\author{M Pavan Manesh\\
EE20MTECH14017 }
\maketitle
\newpage
%\tableofcontents
\bigskip
\renewcommand{\thefigure}{\theenumi}
\renewcommand{\thetable}{\theenumi}
\begin{abstract}
This document explains how to find a row-reduced matrix which is row equivalent to
\end{abstract}
Download all latex-tikz codes from 
%
\begin{lstlisting}
https://github.com/pavanmanesh/EE5609/tree/master/Assignment8
\end{lstlisting}
and all python codes from 
\begin{lstlisting}
https://github.com/pavanmanesh/EE5609/tree/master/Assignment8/codes
\end{lstlisting}
%
\section{Problem}
Find a row-reduced matrix which is row equivalent to,
\begin{align}
\vec{A} = \myvec{1&-i\\2&2\\i& 1+i}\label{A}
\end{align}
What are the solutions of \vec{A}\vec{x}=\vec{0} ?
\section{THEOREM}
\begin{align}
\intertext{Let R be a row-reduced echelon matrix which is row equivalent to A. Then the systems}
A \vec{x} = \vec{0}, R \vec{x} = \vec{0} \label{th}\end{align}
\intertext{have the same solutions.}
\section{Solution}
\textbf{Step 1}: Performing $R_2 \leftarrow R_2-2R_1$ and $R_3 \leftarrow R_3-iR_1$ given by elementary matrix $\vec{E_{31}E_{21}}$ on the equation \eqref{A},
\begin{align}
\vec{E_{31}E_{21}} = \myvec{1&0&0\\-2&1&0\\-i&0&1}\\
\vec{E_{31}E_{21}A} =   \myvec{1&0&0\\-2&1&0\\-i&0&1}\myvec{1&-i\\2&2\\i& 1+i}\\
\implies\vec{A_1} = \vec{E_{31}E_{21}A} = \myvec{1&-i\\0&2+2i\\0&-i}\label{2}
\end{align}
\textbf{Step 2}: Performing $R_2\leftarrow \frac{1}{2+2i}R_2$ given by $\vec{A_1}$ on equation \eqref{2},
\begin{align}
\vec{D_1} = \myvec{1&0&0\\0&\frac{1}{4}(1-i)&0\\0&0&1}\\
\vec{D_1A_1} = \myvec{1&0&0\\0&\frac{1}{4}(1-i)&0\\0&0&1}\myvec{1&-i\\0&2+2i\\0&-i}\\
\implies \vec{A_2}=\vec{D_1A_1} = \myvec{1&-i\\0&1\\0&-i}\label{3}
\end{align}
\textbf{Step 3}: Performing $R_3\leftarrow R_3+iR_2$ given by $\vec{E_{32}}$ on equation \eqref{3},
\begin{align}
\vec{E_{32}} = \myvec{1&0&0\\0&1&0\\0&i&1}
\end{align}
\begin{align}
\vec{E_{32}A_2} = \myvec{1&0&0\\0&1&0\\0&i&1}\myvec{1&i\\0&1\\0&-i}
\end{align}
\begin{align}
\implies \vec{A_3}=\vec{E_{32}A_2} = \myvec{1&-i\\0&1\\0&0}\label{4}
\end{align}
\textbf{Step 4}: Performing $R_1\leftarrow R_1+iR_2$ given by $\vec{E_{12}}$ on equation \eqref{4},
\begin{align}
\vec{E_{12}} = \myvec{1&i&0\\0&1&0\\0&0&1}
\end{align}
\begin{align}
\vec{E_{12}A_3} =\myvec{1&i&0\\0&1&0\\0&0&1} \myvec{1&-i\\0&1\\0&0}
\end{align}
\begin{align}
\implies \vec{A_4} = \vec{E_{12}A_3} = \myvec{1&0\\0&1\\0&0}\label{5}
\end{align}
$\therefore$ Row-reduced matrix of $\vec{A}$ given by equation \eqref{A} is,
\begin{align}
\vec{A} =  \myvec{1&-i\\2&2\\i&1+i}\xleftrightarrow[]{RREF}\myvec{1&0\\0&1\\0&0}\label{final}
\end{align}
From\eqref{th} and \eqref{final},the only solution of  A$\vec{x}=$\vec{0} is,
\begin{align}
    \vec{x}=\myvec{0 \\ 0}
\end{align}
\end{document}
