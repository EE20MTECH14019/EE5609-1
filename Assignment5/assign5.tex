\documentclass[journal,12pt,twocolumn]{IEEEtran}
%
\usepackage{setspace}
\usepackage{gensymb}
%\doublespacing
\singlespacing

%\usepackage{graphicx}
%\usepackage{amssymb}
%\usepackage{relsize}
\usepackage[cmex10]{amsmath}
%\usepackage{amsthm}
%\interdisplaylinepenalty=2500
%\savesymbol{iint}
%\usepackage{txfonts}
%\restoresymbol{TXF}{iint}
%\usepackage{wasysym}
\usepackage{amsthm}
%\usepackage{iithtlc}
\usepackage{mathrsfs}
\usepackage{txfonts}
\usepackage{stfloats}
\usepackage{bm}
\usepackage{cite}
\usepackage{cases}
\usepackage{subfig}
%\usepackage{xtab}
\usepackage{longtable}
\usepackage{multirow}
%\usepackage{algorithm}
%\usepackage{algpseudocode}
\usepackage{enumitem}
\usepackage{mathtools}
\usepackage{steinmetz}
\usepackage{tikz}
\usepackage{circuitikz}
\usepackage{verbatim}
\usepackage{tfrupee}
\usepackage[breaklinks=true]{hyperref}
%\usepackage{stmaryrd}
\usepackage{tkz-euclide} % loads  TikZ and tkz-base
%\usetkzobj{all}
\usetikzlibrary{calc,math}
\usepackage{listings}
    \usepackage{color}                                            %%
    \usepackage{array}                                            %%
    \usepackage{longtable}                                        %%
    \usepackage{calc}                                             %%
    \usepackage{multirow}                                         %%
    \usepackage{hhline}                                           %%
    \usepackage{ifthen}                                           %%
  %optionally (for landscape tables embedded in another document): %%
    \usepackage{lscape}     
\usepackage{multicol}
\usepackage{chngcntr}
%\usepackage{enumerate}

%\usepackage{wasysym}
%\newcounter{MYtempeqncnt}
\DeclareMathOperator*{\Res}{Res}
%\renewcommand{\baselinestretch}{2}
\renewcommand\thesection{\arabic{section}}
\renewcommand\thesubsection{\thesection.\arabic{subsection}}
\renewcommand\thesubsubsection{\thesubsection.\arabic{subsubsection}}

\renewcommand\thesectiondis{\arabic{section}}
\renewcommand\thesubsectiondis{\thesectiondis.\arabic{subsection}}
\renewcommand\thesubsubsectiondis{\thesubsectiondis.\arabic{subsubsection}}

% correct bad hyphenation here
\hyphenation{op-tical net-works semi-conduc-tor}
\def\inputGnumericTable{}                                 %%

\lstset{
%language=C,
frame=single, 
breaklines=true,
columns=fullflexible
}
%\lstset{
%language=tex,
%frame=single, 
%breaklines=true
%}

\begin{document}
%
\newtheorem{theorem}{Theorem}[section]
\newtheorem{problem}{Problem}
\newtheorem{proposition}{Proposition}[section]
\newtheorem{lemma}{Lemma}[section]
\newtheorem{corollary}[theorem]{Corollary}
\newtheorem{example}{Example}[section]
\newtheorem{definition}[problem]{Definition}
%\newtheorem{thm}{Theorem}[section] 
%\newtheorem{defn}[thm]{Definition}
%\newtheorem{algorithm}{Algorithm}[section]
%\newtheorem{cor}{Corollary}
\newcommand{\BEQA}{\begin{eqnarray}}
\newcommand{\EEQA}{\end{eqnarray}}
\newcommand{\define}{\stackrel{\triangle}{=}}
\bibliographystyle{IEEEtran}
%\bibliographystyle{ieeetr}
\providecommand{\mbf}{\mathbf}
\providecommand{\pr}[1]{\ensuremath{\Pr\left(#1\right)}}
\providecommand{\qfunc}[1]{\ensuremath{Q\left(#1\right)}}
\providecommand{\sbrak}[1]{\ensuremath{{}\left[#1\right]}}
\providecommand{\lsbrak}[1]{\ensuremath{{}\left[#1\right.}}
\providecommand{\rsbrak}[1]{\ensuremath{{}\left.#1\right]}}
\providecommand{\brak}[1]{\ensuremath{\left(#1\right)}}
\providecommand{\lbrak}[1]{\ensuremath{\left(#1\right.}}
\providecommand{\rbrak}[1]{\ensuremath{\left.#1\right)}}
\providecommand{\cbrak}[1]{\ensuremath{\left\{#1\right\}}}
\providecommand{\lcbrak}[1]{\ensuremath{\left\{#1\right.}}
\providecommand{\rcbrak}[1]{\ensuremath{\left.#1\right\}}}
\theoremstyle{remark}
\newtheorem{rem}{Remark}
\newcommand{\sgn}{\mathop{\mathrm{sgn}}}
\providecommand{\abs}[1]{\left\vert#1\right\vert}
\providecommand{\res}[1]{\Res\displaylimits_{#1}} 
\providecommand{\norm}[1]{\left\lVert#1\right\rVert}
%\providecommand{\norm}[1]{\lVert#1\rVert}
\providecommand{\mtx}[1]{\mathbf{#1}}
\providecommand{\mean}[1]{E\left[ #1 \right]}
\providecommand{\fourier}{\overset{\mathcal{F}}{ \rightleftharpoons}}
%\providecommand{\hilbert}{\overset{\mathcal{H}}{ \rightleftharpoons}}
\providecommand{\system}{\overset{\mathcal{H}}{ \longleftrightarrow}}
	%\newcommand{\solution}[2]{\textbf{Solution:}{#1}}
\newcommand{\solution}{\noindent \textbf{Solution: }}
\newcommand{\cosec}{\,\text{cosec}\,}
\providecommand{\dec}[2]{\ensuremath{\overset{#1}{\underset{#2}{\gtrless}}}}
\newcommand{\myvec}[1]{\ensuremath{\begin{pmatrix}#1\end{pmatrix}}}
\newcommand{\mydet}[1]{\ensuremath{\begin{vmatrix}#1\end{vmatrix}}}
%\numberwithin{equation}{section}
\numberwithin{equation}{subsection}
%\numberwithin{problem}{section}
%\numberwithin{definition}{section}
\makeatletter
\@addtoreset{figure}{problem}
\makeatother
\let\StandardTheFigure\thefigure
\let\vec\mathbf
%\renewcommand{\thefigure}{\theproblem.\arabic{figure}}
\renewcommand{\thefigure}{\theproblem}
%\setlist[enumerate,1]{before=\renewcommand\theequation{\theenumi.\arabic{equation}}
%\counterwithin{equation}{enumi}
%\renewcommand{\theequation}{\arabic{subsection}.\arabic{equation}}
\def\putbox#1#2#3{\makebox[0in][l]{\makebox[#1][l]{}\raisebox{\baselineskip}[0in][0in]{\raisebox{#2}[0in][0in]{#3}}}}
     \def\rightbox#1{\makebox[0in][r]{#1}}
     \def\centbox#1{\makebox[0in]{#1}}
     \def\topbox#1{\raisebox{-\baselineskip}[0in][0in]{#1}}
     \def\midbox#1{\raisebox{-0.5\baselineskip}[0in][0in]{#1}}
\vspace{3cm}
\title{EE5609: Matrix Theory\\
          Assignment-5\\}
\author{M Pavan Manesh\\
EE20MTECH14017 }
\maketitle
\newpage
%\tableofcontents
\bigskip
\renewcommand{\thefigure}{\theenumi}
\renewcommand{\thetable}{\theenumi}
\begin{abstract}
This document contains solution to determine the conic representing the given equation. 
\end{abstract}
Download the python codes from 
%
%
latex-tikz codes from 
%
\begin{lstlisting}
https://github.com/pavanmanesh/EE5609/tree/master/Assignment5
\end{lstlisting}
%
\section{Problem}
What conic does the following equation represent. 
\begin{align*}
y^2-2\sqrt{3}xy+3x^2+6x-4y+5 = 0
\end{align*}
Find the center.
\section{Solution}
The general second degree equation can be expressed as follows,
\begin{align}
\vec{x^T}\vec{V}\vec{x}+2\vec{u^T}\vec{x}+f=0\label{eqmain}
\end{align}
From the given second degree equation we get,
\begin{align}
\vec{V} &= \myvec{3&-\sqrt{3}\\-\sqrt{3}&1}\\ \label{given1}
\vec{u} &= \myvec{3\\-2}\\ 
f &= 5 \label{given2}
\end{align}
Expanding the determinant of $\vec{V}$ we observe, 
\begin{align}
\mydet{3&-\sqrt{3}\\-\sqrt{3}&1} = 0 \label{eq2.1}
\end{align}
Also
\begin{align}
    \mydet{\vec{V} & \vec{u} \\ \vec{u}^T & f}=
    \mydet{3 &-\sqrt{3} & 3 \\ -\sqrt{3}&1 & -2 \\ 3 & -2 & 5} \\
    = 12\sqrt{3}-21\neq 0\label{eq2.2}
\end{align}
Hence from \eqref{eq2.1} and \eqref{eq2.2} we conclude that given equation is an parabola. The characteristic equation of $\vec{V}$ is given as follows,
\begin{align}
\mydet{\lambda\vec{I}-\vec{V}} = \mydet{\lambda-3&\sqrt{3}\\\sqrt{3}&\lambda-1} &= 0\\
\implies \lambda^2-4\lambda &= 0\label{eqchar}
\end{align}
Hence the characteristic equation of $\vec{V}$ is given by \eqref{eqchar}. The roots of \eqref{eqchar} i.e the eigenvalues are given by
\begin{align}
\lambda_1=0, \lambda_2=4\label{eqeigenvals}    
\end{align}
The eigen vector $\vec{p}$ is defined as, 
\begin{align}
\vec{V}\vec{p} &= \lambda\vec{p}\\
\implies\brak{\lambda\vec{I}-\vec{V}}\vec{p}&=0
\end{align}
for $\lambda_1=0$,
\begin{align}
\brak{\lambda_1\vec{I}-\vec{V}}&=\myvec{-3&\sqrt{3}\\\sqrt{3}&-1}\xleftrightarrow[R_1=\frac{1}{3}R_1]{R_2=\sqrt{3}R_2+R_1}\myvec{-1&\frac{1}{\sqrt{3}}\\0&0}\\
\implies\vec{p_1}&=\myvec{\frac{1}{\sqrt{3}}\\1} \label{eq2.3}
\end{align}
Again, for $\lambda_2=4$,
\begin{align}
\brak{\lambda_2\vec{I}-\vec{V}}&=\myvec{1&\sqrt{3}\\\sqrt{3}&3}\xleftrightarrow[R_2=\frac{R_2}{\sqrt{3}}R_1]{R_1=R_1}\myvec{1&\sqrt{3}\\0&0}\\
\implies\vec{p_2}=\myvec{\sqrt{3}\\-1}
\end{align}
The matrix \vec{P},
\begin{align}
\vec{P}&=\myvec{\vec{p_1}&\vec{p_2}}=\myvec{\frac{1}{\sqrt{3}}&\sqrt{3}\\1&-1} \\
\vec{D}&=\myvec{0&0\\0&4}
\end{align}
\begin{align}
    \eta=2\vec{u}^T\vec{p_1}=2(\sqrt{3}-2)
\end{align}
The focal length of the parabola is given by:
\begin{align}
    \abs{\frac{\eta}{\lambda_2}} 
    = \frac{2-\sqrt{3}}{2}
\end{align}
When $\mydet{\vec{V}}=0$,\eqref{eqmain} can be written as
\begin{align}
    \vec{y^T}\vec{D}\vec{y}&=\eta\myvec{1&0}\vec{y}\label{eq2.4}
    \intertext{And the vertex $\vec{c}$ is given by }
    \myvec{\vec{u^T}+\eta\vec{p_1^T} \\ \vec{V}}\vec{c}=
    \myvec{-f \\ \eta\vec{p_1}-\vec{u}} 
\end{align}
using equations \eqref{given1},\eqref{given2} and \eqref{eq2.3}
\begin{align}
    \myvec{5-\frac{4}{\sqrt{3}}& 2\sqrt{3}-6 \\ 3 & -\sqrt{3} \\  -\sqrt{3} & 1 }\vec{c}=\myvec{-5 \\ -1-\frac{4}{\sqrt{3}} \\ 
    2\sqrt{3}-2 }
\end{align}
\begin{align}
 \myvec{5-\frac{4}{\sqrt{3}}& 2\sqrt{3}-6 & -5\\ 3 & -\sqrt{3} & -1-\frac{4}{\sqrt{3}}\\  -\sqrt{3} & 1 & 2\sqrt{3}-2 } 
\xleftrightarrow[R_1=\frac{R_1}{5-\frac{4}{\sqrt{3}}}]{R_2=R_2-3R_1} \nonumber \\
 \myvec{1& \frac{-66+6\sqrt{3}}{59} & \frac{-75-20\sqrt{3}}{59} \\ 0 &\frac{198-77\sqrt{3}}{59} & \frac{166}{59}-\frac{-56\sqrt{3}}{177}\\  -\sqrt{3} & 1 & 2\sqrt{3}-2 }  
 \xleftrightarrow[R_2=\frac{R_2}{\frac{198-77\sqrt{3}}{59}}]{R_3=R_3+\sqrt{3}R_1} \nonumber \\
 \myvec{1& \frac{-66+6\sqrt{3}}{59} & \frac{-75-20\sqrt{3}}{59} \\
 0 &1 & \frac{14\sqrt{3}+44}{33} \\
 0 & \frac{77-66\sqrt{3}}{59} & \frac{-178+43\sqrt{3}}{59}}  \xleftrightarrow[R_1=R_1-\frac{-66+6\sqrt{3}}R_2]{R_3=R_3-\frac{77-66\sqrt{3}}{59}R_2} \nonumber \\
  \myvec{1& 0 & \frac{1}{11} \\
 0 &1 & \frac{14\sqrt{3}+44}{33} \\
 0 & 0 & \frac{-10+5\sqrt{3}}{3}} 
\end{align}
\end{document}
